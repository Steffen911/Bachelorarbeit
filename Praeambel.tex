% Hinweis: Optionen der Dokumentenklasse werden an alle folgenden \usepackage{package} Befehle weitergegeben
\documentclass[
	pdftex,
	fontsize=12pt,
	paper=a4,
	halfparskip,
	twoside=false,
	numbers=noenddot,	% Kein Punkt am Ende einer Überschrift
	%draft=true,			% Deckt Schwächen auf: overfull und full boxes werden markiert; Bilder werden nicht geladen
	bibliography=totoc,	% Literaturverzeichnis ins Inhaltsverzeichnis aufnehmen
	listof=totoc,		% Tabellen- und Abbildungsverzeichnis ins Inhaltsverzeichnis aufnehmen
	titlepage=true,		% Separate Titelseite; Gestaltung mit Hilfe der Titlepage-Umgebung
	headsepline=true,	% Kopflinie aktivieren
	footsepline=true	,	% Fußlinie aktivieren
	abstracton			% Abstract aktivieren
]{scrreprt}

% Zeichenkodierung Input ist UTF-8: Umlaute können direkt eingegeben werden
\usepackage[utf8]{inputenc}

\usepackage{tabularx}

% Zeichenkodierung Ausgabe ist T1-Kodierung: Wichtig für die Ausgabe von Umlauten
\usepackage[T1]{fontenc}

% Schrift festlegen
\usepackage{lmodern}

% Sprachauswahl für Lokalisierungen und Silbentrennung
\usepackage[ngerman]{babel}

% Zitate: Anführungszeichen automatisch anhand der Sprache wählen
\usepackage[babel=true]{csquotes}

% Source-Code-Listings
\usepackage{listings}

% BibTeX-Symbol
\usepackage{texnames}

\usepackage{xcolor}

% Symbole, z.B. Haken
\usepackage{pifont}

% Zeilen in Tabellen zusammenfassen
\usepackage{multirow}

% Silbentrennung kann bei bestimmten Wörten mit Hilfe von diesem Paket deaktiviert werden
\usepackage{hyphenat}

% Abkürzungsverzeichnis
\usepackage[printonlyused, withpage]{acronym}
% Abstand mit Punkten füllen
%\renewcommand*\bflabel[1]{\textbf{\normalsize{#1}}\hfill}

% Tiefe des Inhaltsverzeichnisses
\setcounter{tocdepth}{1}

% Punkte im Inhaltsverzeichnis
\usepackage{tocstyle}
\usetocstyle{allwithdot}

% Zum Einbinden von PDF-Dateien.
\usepackage{pdfpages}

% Paket zum Anpassen von Kopf- und Fußzeilen
\usepackage[plainfootsepline, plainheadsepline, headsepline, footsepline, automark]{scrpage2}
% Liniendicke
\setheadsepline{0.1pt}
\setfootsepline{0.1pt}

% Kopf- und Fusszeile löschen
\clearscrheadfoot
% Kopf- und Fusszeile aktivieren
\pagestyle{scrheadings}

% Kopf links
\ihead[\titel]{\titel}

% Fuss links
\ifoot[\verfasser]{\verfasser}
% Fuss rechts
\ofoot[\pagemark]{\pagemark}

% Grafiken einbinden
\usepackage{graphicx}
% Pfad zu den Grafiken
\graphicspath{{Abbildungen/}}

% Seitenränder setzen
\usepackage[left=2cm, right=4cm, top=2.5cm, bottom=2cm]{geometry}

% Zeilenabstand auf 1.5 setzen
\usepackage{setspace}
\onehalfspacing

% Literaturverzeichnis
%\usepackage[backend=bibtex,style=authoryear]{biblatex}
\usepackage{bibgerm}
\usepackage[backend=bibtex,style=alphabetic]{biblatex}
%\bibliographystyle{alphadin}

\bibliography{Literatur/Literatur.bib}

% Glossar
\usepackage[toc,nonumberlist]{glossaries}

% Titel als Referenzierung verwenden
\usepackage{titleref}

% Währungen
\usepackage{textcomp}

% Fussnoten fortlaufend nummerieren.
\usepackage{chngcntr}
\counterwithout{footnote}{chapter}

% Persönliche Daten
\newcommand{\titel}{Qualitätssicherung in agilen Softwareprojekten am Beispiel eines Softwareprojektes bei einer Bausparkasse}
\newcommand{\untertitel}{}
\newcommand{\art}{Bachelorarbeit}
\newcommand{\verfasser}{Steffen Schmitz}
\newcommand{\kurs}{WWI13SCA}
\newcommand{\ausbildungsbetrieb}{SAP SE}%, Dietmar-Hopp-Allee 16, 69190 Walldorf}
\newcommand{\ausbildungsadresse}{Dietmar-Hopp-Allee 16}
\newcommand{\ausbildungsort}{69190 Walldorf}
\newcommand{\abgabedatum}{15. Februar 2016}

\newcommand{\martrikelnr}{4864651}
\newcommand{\betreuer}{Karlheinz Natt}
\newcommand{\betreuermail}{kxxxxxxx@xxxx.com}
\newcommand{\studiengangsleiter}{Frank Koslowski}
\newcommand{\zeitraum}{20.11.2015 - 15.02.2016}


% Links- und PDF-Einstellungen
\usepackage{hyperref}
\hypersetup{
	pdfauthor = {\verfasser},
	pdftitle = {\titel},
	pdfsubject = {\art},
	pdfkeywords = {},
	pdfstartview = {Fit},
	colorlinks = {false},
	breaklinks = {true},
	bookmarksopen = {true}
}

% Verhinderung von Schusterjunge und Hurenkind
\clubpenalty = 10000
\widowpenalty = 10000
\displaywidowpenalty = 10000

% Seitenzähler für große, römische Zahlen
\newcounter{RomanPagenumber}

% Abkürzungen
\newcommand{\dash}{d.\,h.}
\newcommand{\zB}{z.\,B.}

% Mathematiksymbole
\usepackage{amssymb}
\usepackage{amsthm}

%Fix position of images
\usepackage{float} 
