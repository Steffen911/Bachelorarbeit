\chapter{Fazit}

    \section{Zusammenfassung}
    
        Das Ziel dieser Arbeit ist es, die Schwachstellen der agilen Entwicklungsverfahren im Hinblick auf die Sicherung einer vorher festgelegten Qualität zu gewährleisten und Mittel aufzuzeigen, wie diese Verfahren optimiert werden können. Außerdem soll ein Prozess skizziert werden, um die Ergebnisse dieser Arbeit in ein Projekt zu integrieren und die Ergebnisse der Iterationen ständig zu verbessern.
        
        Hierzu wurden Qualitätsmerkmale herausgearbeitet, die verfolgt werden sollen und in die Grundlagen der agilen Methoden und der Qualitätssicherungsmethoden eingeführt.
        
        Es wurde geprüft, welche Teile der Qualitätsmerkmale bereits durch agile Methoden abgedeckt sind. Der Scrum-Prozess wurde als Grundlage verwendet und, falls nötig, durch Methoden des Extreme Programming und Lean Software Development erweitert. Hieraus ergaben sich Unterdimensionen der Qualitätsmerkmale, die nicht durch agile Methoden abgedeckt sind und weiterhin betrachtet werden.
        
        Für diese kritischen Unterdimensionen wurden die zuvor vorgestellten Qualitätsmanagementsysteme analysiert, um Optimierungsvorschläge zu geben. Durch die Kombination der agilen Verfahren und der Qualitätsmanagementsysteme entstand eine Menge von Methoden und Hinweisen, die die Qualität des Softwareentwicklungsprozesses verbessern.
        
        Zum Abschluss wurde auf Grundlage der ISO 9000 eine Vorgehensweise skizziert, wie die aufgezeigten Methoden in das Projekt integriert werden können. Hierzu wird der Deming-Kreis verwendet und mit jeder Iteration werden neue Methoden eingeführt und das Ergebnis überprüft.
    
    \section{Kritische Bewertung und Ausblick}
    
        Die Ergänzungen zu agilen Entwicklungsmethoden, wie sie in dieser Arbeit vorgeschlagen werden, stellen mögliche Ansätze dar, um Qualität in agilen Methoden zu gewährleisten. Sie stellen keine umfangreiche Lösung dar, sondern stellen Ansätze bereit. Da diese Vorschläge auf der Grundlage von Literaturarbeit entwickelt wurden, müssen sie sich in der Praxis zunächst behaupten.
        
        Zum Einsatz in einem Softwareprojekt müssen die vorgeschlagenen Ansätze empirisch belegt werden. Hierzu können sie in Projekten getestet werden, oder durch Interviews und Befragungen von Experten validiert werden. Nachdem diese als eine Lösung der vorgestellten Herausforderung bestätigt sind, können sie, wie in dieser Arbeit skizziert, dauerhaft in ein Projekt integriert werden.
        
        Abschließend ist festzustellen, dass kein perfekter Qualitätsansatz existiert, da in jedem Prozess Verbesserungspotenzial besteht.