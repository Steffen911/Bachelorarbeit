\addchap{Kurzfassung}

%    \begin{tabularx}{\textwidth}{lX}
%     \textbf{Verfasser} & Steffen Schmitz \\
%      \textbf{Kurs} & WWI13SCA \\
%      \textbf{Firma} & SAP SE \\
%      \textbf{Thema} & Qualitätssicherung in agilen Softwareprojekten am Besipiel eines Softwareprojektes bei einer Bausparkasse \\
%    \end{tabularx}

    \vspace{\baselineskip}

    Die Entwicklung von funktionaler, verlässlicher und wartbarer Software ist für Bausparkassen ein komplexes Verfahren. Mit klassischen Softwareentwicklungsmethoden werden die Zeit- und Budgetpläne für Projekte häufig überzogen und die Software wird nicht, oder nicht vollständig, den zuvor definierten Qualitätsanforderungen gerecht.

    Agile Entwicklungsmethoden, wie Scrum und Extreme Programming, wurden entwickelt, um im Rahmen des vorgegeneben Zeit- und Budgetplans eine hohe Qualität der Software zu gewährleisten. Durch flexible Entwicklungspläne und kurze Iterationen, die jeweils fertige, potenziell auslieferbare Software produzieren, sind diese Projekte sehr dynamisch und bieten gegenüber dem Zeit- und Budgetaspekt eine sehr gute Planbarkeit.

    Eine häufige Sorge ist es, dass die Qualität der Software auf Grund der kurzen Entwicklungszeiten und des kurzen Planungshorizonts nicht sichergestellt werden kann.

    Im Rahmen dieser Arbeit werden Erweiterungen für die agilen Methoden entwickelt, die die Qualität erhöhen. Hierzu werden Lean Software Development, Scrum und Extreme Programming kombiniert und um Methoden aus anerkannten Qualitätssicherungsstandards wie ISO 9000 und Six Sigma erweitert. Das Ergebnis sind Ansätze, wie bestimmte Qualitätskriterien auch in agilen Projekten erfüllt werden können.

    Darauf aufbauend wird, auf Basis der ISO 9000, ein Prozess skizziert, wie die Ansätze in agile Projekte eingebunden werden können. Der Mehrwert dieser Maßnahmen wird im Rahmen des skizzierten Prozesses überprüft und bewiesen. Um die Vorgehensweise des agilen Projektes kontinuierlich zu verbessern, muss der Prozess in jeder Iteration durchlaufen werden. 
